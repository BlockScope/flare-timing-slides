\documentclass{beamer}
\usetheme{Warsaw}
\usepackage{gap-civl}

\title{Flare Timing}
\subtitle{Scoring Free Flight Competitions}
\author{Phil de Joux}
\institute{Block Scope}
\date{}

\begin{document}

\begin{frame}
\titlepage
\end{frame}

\begin{frame}{Race Task}
    Free Flight Competitions for Hang gliding or Paragliding
\begin{itemize}
    \item A race task each day of a comp
    \item Like a yachting buoy race in the sky
    \item Game of tag each turnpoint\footnote{Turnpoint are vertical cylinders} and make goal
    \item Flown by gliding in straight lines and circling in lift
\end{itemize}
\end{frame}


\begin{frame}{Rules}
The GAP rules were originally devised by \textbf{G}erolf Heinrichs, \textbf{A}ngelo Crapanzano and \textbf{P}aul Mollison.
\[ Total = DistancePoints + TimePoints + LeadingPoints + ArrivalPoints \]
    
\begin{itemize}
    \item CIVL\footnote{Commission Internationale de Vol Libre} makes the rules
    \item scorers apply the rules
    \item pilots contest the rules
\end{itemize}
\end{frame}

\begin{frame}{Scoring Process}
\begin{figure}[!ht]
    \centering
    \begin{tikzpicture}[node distance=2cm,scale=0.4, every node/.style={scale=0.4}]

\node (task) [io, xshift=-6cm] {Define task};
\node (fly) [process, below of=task] {Fly task};
\node (track) [decision, below of=fly] {Evaluate track logs};
\node (valid) [decision, below of=track] {Calculate task validity};
\node (points) [decision, below of=valid] {Allocate available points};
\node (score) [decision, below of=points] {Score flights};

\draw [arrow] (task) -- (fly);
\draw [arrow] (fly) -- (track);
\draw [arrow] (track) -- (valid);
\draw [arrow] (valid) -- (points);
\draw [arrow] (points) -- (score);

\end{tikzpicture}

    \label{fig:scoring-process}
\end{figure}
\end{frame}

\begin{frame}{Why Flare Timing?}
FS, the official scoring program, is a tangled spaghetti of data access, scoring math and user interface.
\begin{itemize}
    \item Only the scoring math
    \item Show more of the working
    \item Verify that scores match the spec', the GAP rules
    \item Compare computed results
\end{itemize}
\end{frame}

\begin{frame}{Why Haskell?}
\begin{itemize}
    \item Sum types and exhaustive pattern matching
    \item Property testing with \texttt{quickcheck}
    \item Units of measure with \texttt{uom-plugin}
    \item Compare floating point and rational math
    \item Package \texttt{fgl} for shortest path
    \item Parser combinators for file format parsing
\end{itemize}
\end{frame}

\end{document}
